\documentclass[10pt]{article} % Default font size

 
\usepackage{hyperref}
\hypersetup{
    colorlinks=true,
    linkcolor=blue,
    filecolor=magenta,      
    urlcolor=cyan,
}

\thispagestyle{empty}
 
\urlstyle{same}

\input{structure.tex} % Include the file specifying document layout
%----------------------------------------------------------------------------------------

\begin{document}

%----------------------------------------------------------------------------------------
%	NAME AND CONTACT INFORMATION
%----------------------------------------------------------------------------------------

\title{Куат Абылкасымов} % Print the main header

%------------------------------------------------

\parbox{0.5\textwidth}{ % First block
\begin{tabbing} % Enables tabbing
\hspace{3cm} \= \hspace{4cm} \= \kill % Spacing within the block
% {\bf Address} \> Zakharova 50A, 18\\ % Address line 1
% \> Almaty \\ % Address line 2
{\bf Возраст: } \> 22 года \\ % Date of birth 
{\bf Github:} \> \href{https://www.github.com/kuator}{www.github.com/kuator} \\ % Github
\end{tabbing}}
\hfill % Horizontal space between the two blocks
\parbox{0.5\textwidth}{ % Second block
\begin{tabbing} % Enables tabbing
\hspace{3cm} \= \hspace{4cm} \= \kill % Spacing within the block
{\bf Номер: } \> +7 (708) 284 3988 \\ % Mobile phone
{\bf Почта: } \> \href{mailto:kuatabylkasymov@gmail.com}{kuatabylkasymov@gmail.com} \\ % Email address
{\bf Сайт-визитка: } \> \href{https://kuator.github.io}{kuator.github.io} \\ % Site
\end{tabbing}}

%----------------------------------------------------------------------------------------
%	EDUCATION SECTION
%----------------------------------------------------------------------------------------

\section{Образование}

\tabbedblock{
\bf{2016-2020} \> Бакалавр специальности "Информационные системы" - \href{http://www.kbtu.kz}{КБТУ} \\[5pt]
\>GPA: 3.5\\
}

%----------------------------------------------------------------------------------------
%	EMPLOYMENT SECTION
%----------------------------------------------------------------------------------------
\section{Опыт работы}

\job
{Апр 2021-}{Май 2021}
{\textbf{Место работы: }TOO "Control-Link" }
{\textbf{Позиция}: Техник-программист. Tехнологии: PLC, OPCUA}
\normalfont{\textbf{Обязанности}: Сбор и обработка данных с сервера OPCUA}

\job
{Янв 2020-}{Окт 2020}
{\textbf{Место работы: }TOO "CodeBusters" }
{\textbf{Позиция}: Бэкенд разработчик. Tехнологии: Django, DRF, Docker}
\normalfont{\textbf{Обязанности}: Разработка платформы для школы-интерната IQanat}


%------------------------------------------------

%----------------------------------------------------------------------------------------
%	IT/COMPUTING SKILLS SECTION
%----------------------------------------------------------------------------------------

\section{Персональные работы}


\href{https://github.com/kuator/chachachat}{Chachachat}: Чат, построенный на Django и ReactJS. \\
Для создания сокетов использовал django-channels на бэкенде, и WebSockets API на фронтэнде.\\
Также использовал async await для неблокирующих вызовов. На фронтэнде используются React Hooks и Tailwindcss\\

\href{https://github.com/kuator/customer-relations-management}{CRM}: CRM-система для мэнэджемента лидов и агентов с разделением на роли.\\
В этом проекте использовал встроенный в Django движок шаблонов Jinja. \\

\href{https://github.com/kuator/tcp-server}{TCP-server}: TCP server на C для лучшего понимания как работают сокеты,\\
в чем разница между unix-сокетами и network-сокетами.


\section{Навыки}

\skillgroup{Web}
{
  Backend: Django(Django Rest Framework, Django Channels, Celery), Fastapi, Node.js(Express)\\
  Frontend: React(React Hooks, Redux), CSS(Tailwindcss)
}

\skillgroup{Языки программирования}
{
  Python: asyncio, threading, GIL\\
  Javascript: ES5, ES6\\
  C++: для решения задач на Hackerrank\\
  Lua: для скриптинга и для конфигураций
}

\skillgroup{Другие навыки}
{
  Linux: продвинутый пользователь\\
  Docker: знакомство с базовыми командами\\
  Git: знание на уровне решения merge conflict-ов\\
  Databases: Postgresql, MongoDB
}

%------------------------------------------------

\skillgroup{Среда разработки}
{
  OS: Linux Mint 20\\
  SHELL: zsh\\
  IDE: Vim
}

\skillgroup{О себе}
{
  Беру от жизни всё, девушка в супермаркете предложила \\ 
  мне кусочек колбасы для дегустации, я взял два. \\ 
  Люблю CSS анимации: \href{https://kuator.github.io/gh-pages/image-reveal-kinda}{ToothBrush},
    \href{https://kuator.github.io/gh-pages/cloudmoon}{Cloudmoon},
    \href{https://kuator.github.io/gh-pages/rain/}{Rain}
}

\end{document}
