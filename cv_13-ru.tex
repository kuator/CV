\documentclass[10pt]{article} % Default font size

 
\usepackage{hyperref}
\hypersetup{
    colorlinks=true,
    linkcolor=blue,
    filecolor=magenta,      
    urlcolor=cyan,
}

\thispagestyle{empty}
 
\urlstyle{same}

%%%%%%%%%%%%%%%%%%%%%%%%%%%%%%%%%%%%%%%%%
% Wilson Resume/CV
% Structure Specification File
% Version 1.0 (22/1/2015)
%
% This file has been downloaded from:
% http://www.LaTeXTemplates.com
%
% License:
% CC BY-NC-SA 3.0 (http://creativecommons.org/licenses/by-nc-sa/3.0/)
%
%%%%%%%%%%%%%%%%%%%%%%%%%%%%%%%%%%%%%%%%%

%----------------------------------------------------------------------------------------
%	PACKAGES AND OTHER DOCUMENT CONFIGURATIONS
%----------------------------------------------------------------------------------------

\usepackage[a4paper, hmargin=25mm, vmargin=30mm, top=20mm]{geometry} % Use A4 paper and set margins

\usepackage{fancyhdr} % Customize the header and footer

\usepackage{lastpage} % Required for calculating the number of pages in the document

\usepackage{hyperref} % Colors for links, text and headings

\setcounter{secnumdepth}{0} % Suppress section numbering

%\usepackage[proportional,scaled=1.064]{erewhon} % Use the Erewhon font
%\usepackage[erewhon,vvarbb,bigdelims]{newtxmath} % Use the Erewhon font
\usepackage[utf8]{inputenc} % Required for inputting international characters
\usepackage[T1]{fontenc} % Output font encoding for international characters

\usepackage{fontspec} % Required for specification of custom fonts
\setmainfont[Path = ./fonts/,
Extension = .otf,
BoldFont = Erewhon-Bold,
ItalicFont = Erewhon-Italic,
BoldItalicFont = Erewhon-BoldItalic,
SmallCapsFeatures = {Letters = SmallCaps}
]{Erewhon-Regular}

\usepackage{color} % Required for custom colors
\definecolor{slateblue}{rgb}{0.17,0.22,0.34}

\usepackage{sectsty} % Allows customization of titles
\sectionfont{\color{slateblue}} % Color section titles

\fancypagestyle{plain}{\fancyhf{}\cfoot{\thepage\ of \pageref{LastPage}}} % Define a custom page style
\pagestyle{plain} % Use the custom page style through the document
\renewcommand{\headrulewidth}{0pt} % Disable the default header rule
\renewcommand{\footrulewidth}{0pt} % Disable the default footer rule

\setlength\parindent{0pt} % Stop paragraph indentation

% Non-indenting itemize
\newenvironment{itemize-noindent}
{\setlength{\leftmargini}{0em}\begin{itemize}}
{\end{itemize}}

% Text width for tabbing environments
\newlength{\smallertextwidth}
\setlength{\smallertextwidth}{\textwidth}
\addtolength{\smallertextwidth}{-2cm}

\newcommand{\sqbullet}{~\vrule height 1ex width .8ex depth -.2ex} % Custom square bullet point definition

%----------------------------------------------------------------------------------------
%	MAIN HEADER COMMAND
%----------------------------------------------------------------------------------------

\renewcommand{\title}[1]{
{\huge{\color{slateblue}\textbf{#1}}}\\ % Header section name and color
\rule{\textwidth}{0.5mm}\\ % Rule under the header
}

%----------------------------------------------------------------------------------------
%	JOB COMMAND
%----------------------------------------------------------------------------------------

\newcommand{\job}[6]{
\begin{tabbing}
\hspace{2cm} \= \kill
\textbf{#1} \> \normalfont{#3} \\
\textbf{#2} \>\+ \normalfont{#4} \\
\begin{minipage}{\smallertextwidth}
\vspace{2mm}
#6
\end{minipage}
\end{tabbing}
\vspace{2mm}
}

%----------------------------------------------------------------------------------------
%	SKILL GROUP COMMAND
%----------------------------------------------------------------------------------------

\newcommand{\skillgroup}[2]{
\begin{tabbing}
\hspace{5mm} \= \kill
\sqbullet \>\+ \textbf{#1} \\
\begin{minipage}{\smallertextwidth}
\vspace{2mm}
#2
\end{minipage}
\end{tabbing}
}

%----------------------------------------------------------------------------------------
%	INTERESTS GROUP COMMAND
%-----------------------------------------------------------------------------------------

\newcommand{\interestsgroup}[1]{
\begin{tabbing}
\hspace{5mm} \= \kill
#1
\end{tabbing}
\vspace{-10mm}
}

\newcommand{\interest}[1]{\sqbullet \> \textbf{#1}\\[3pt]} % Define a custom command for individual interests

%----------------------------------------------------------------------------------------
%	TABBED BLOCK COMMAND
%----------------------------------------------------------------------------------------

\newcommand{\tabbedblock}[1]{
\begin{tabbing}
\hspace{2cm} \= \hspace{4cm} \= \kill
#1
\end{tabbing}
}
 % Include the file specifying document layout
%----------------------------------------------------------------------------------------

\begin{document}

%----------------------------------------------------------------------------------------
%	NAME AND CONTACT INFORMATION
%----------------------------------------------------------------------------------------

\title{Куат Абылкасымов} % Print the main header

%------------------------------------------------

\parbox{0.5\textwidth}{ % First block
\begin{tabbing} % Enables tabbing
\hspace{3cm} \= \hspace{4cm} \= \kill % Spacing within the block
% {\bf Address} \> Zakharova 50A, 18\\ % Address line 1
% \> Almaty \\ % Address line 2
{\bf Возраст: } \> 22 года \\ % Date of birth 
{\bf Github:} \> \href{https://www.github.com/kuator}{www.github.com/kuator} \\ % Github
\end{tabbing}}
\hfill % Horizontal space between the two blocks
\parbox{0.5\textwidth}{ % Second block
\begin{tabbing} % Enables tabbing
\hspace{3cm} \= \hspace{4cm} \= \kill % Spacing within the block
{\bf Номер: } \> +7 (708) 284 3988 \\ % Mobile phone
{\bf Почта: } \> \href{mailto:kuatabylkasymov@gmail.com}{kuatabylkasymov@gmail.com} \\ % Email address
{\bf Сайт-визитка: } \> \href{https://kuator.github.io}{kuator.github.io} \\ % Site
\end{tabbing}}

%----------------------------------------------------------------------------------------
%	EDUCATION SECTION
%----------------------------------------------------------------------------------------

\section{Образование}

\tabbedblock{
\bf{2016-2020} \> Бакалавр специальности "Информационные системы" - \href{http://www.kbtu.kz}{КБТУ} \\[5pt]
\>GPA: 3.5\\
}

%----------------------------------------------------------------------------------------
%	EMPLOYMENT SECTION
%----------------------------------------------------------------------------------------
\section{Опыт работы}

\job
{Апр 2021-}{Май 2021}
{\textbf{Место работы: }TOO "Control-Link" }
{\textbf{Позиция}: Техник-программист. Tехнологии: PLC, OPCUA}
\normalfont{\textbf{Обязанности}: Сбор и обработка данных с сервера OPCUA}

\job
{Янв 2020-}{Окт 2020}
{\textbf{Место работы: }TOO "CodeBusters" }
{\textbf{Позиция}: Бэкенд разработчик. Tехнологии: Django, DRF, Docker}
\normalfont{\textbf{Обязанности}: Разработка платформы для школы-интерната IQanat}


%------------------------------------------------

%----------------------------------------------------------------------------------------
%	IT/COMPUTING SKILLS SECTION
%----------------------------------------------------------------------------------------

\section{Персональные работы}


\href{https://github.com/kuator/chachachat}{Chachachat}: Чат, построенный на Django и ReactJS. \\
Для создания сокетов использовал django-channels на бэкенде, и WebSockets API на фронтэнде.\\
Также использовал async await для неблокирующих вызовов. На фронтэнде используются React Hooks и Tailwindcss\\

\href{https://github.com/kuator/customer-relations-management}{CRM}: CRM-система для мэнэджемента лидов и агентов с разделением на роли.\\
В этом проекте использовал встроенный в Django движок шаблонов Jinja. \\

\href{https://github.com/kuator/tcp-server}{TCP-server}: TCP server на C для лучшего понимания как работают сокеты,\\
в чем разница между unix-сокетами и network-сокетами.


\section{Навыки}

\skillgroup{Web}
{
  Backend: Django(Django Rest Framework, Django Channels, Celery), Fastapi, Node.js(Express)\\
  Frontend: React(React Hooks, Redux), CSS(Tailwindcss)
}

\skillgroup{Языки программирования}
{
  Python: asyncio, threading, GIL\\
  Javascript: ES5, ES6\\
  C++: для решения задач на Hackerrank\\
  Lua: для скриптинга и для конфигураций
}

\skillgroup{Другие навыки}
{
  Linux: продвинутый пользователь, использую базовые утилиты как awk, grep, find, pipes\\
  Docker: знакомство с базовыми командами\\
  Git: знание на уровне решения merge conflict-ов\\
  Databases: Postgresql, MongoDB
}

%------------------------------------------------

\skillgroup{Среда разработки}
{
  OS: Linux Mint 20\\
  SHELL: zsh\\
  IDE: Vim
}

\skillgroup{О себе}
{
  Беру от жизни всё, девушка в супермаркете предложила \\ 
  мне кусочек колбасы для дегустации, я взял два. \\ 
  Люблю CSS анимации: \href{https://kuator.github.io/gh-pages/image-reveal-kinda}{ToothBrush},
    \href{https://kuator.github.io/gh-pages/cloudmoon}{Cloudmoon},
    \href{https://kuator.github.io/gh-pages/rain/}{Rain}
}

\end{document}
