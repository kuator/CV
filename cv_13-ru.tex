\documentclass[10pt]{article} % Default font size

 
\usepackage{hyperref}
\hypersetup{
    colorlinks=true,
    linkcolor=blue,
    filecolor=magenta,      
    urlcolor=blue,
}

\thispagestyle{empty}
 
\urlstyle{same}

\input{structure.tex} % Include the file specifying document layout
%----------------------------------------------------------------------------------------

\begin{document}

%----------------------------------------------------------------------------------------
%	NAME AND CONTACT INFORMATION
%----------------------------------------------------------------------------------------

\title{Куат Абылкасымов} % Print the main header

%------------------------------------------------

\parbox{0.5\textwidth}{ % First block
\begin{tabbing} % Enables tabbing
\hspace{3cm} \= \hspace{4cm} \= \kill % Spacing within the block
% {\bf Address} \> Zakharova 50A, 18\\ % Address line 1
% \> Almaty \\ % Address line 2
{\bf Возраст: } \> 25 лет \\ % Date of birth 
{\bf Github:} \> \href{https://www.github.com/kuator}{github.com/kuator} \\ % Github
{\bf Linkedin:} \> \href{https://www.linkedin.com/in/kuat-abylkassymov-273bb2177/}{linke.din/kuat} \\ % Linkedin
\end{tabbing}}
\hfill % Horizontal space between the two blocks
\parbox{0.5\textwidth}{ % Second block
\begin{tabbing} % Enables tabbing
\hspace{3cm} \= \hspace{4cm} \= \kill % Spacing within the block
{\bf Номер: } \> +7 (708) 284 3988 \\ % Mobile phone
{\bf Почта: } \> \href{mailto:kuatabylkasymov@gmail.com}{kuatabylkasymov@gmail.com} \\ % Email address
{\bf Сайт-визитка: } \> \href{https://kuator.github.io}{kuator.github.io} \\ % Site
\end{tabbing}}

%----------------------------------------------------------------------------------------
%	EDUCATION SECTION
%----------------------------------------------------------------------------------------

\section{Образование}

\tabbedblock{
\bf{2016-2020} \> Бакалавр специальности "Информационные системы" - \href{http://www.kbtu.kz}{КБТУ} }

%----------------------------------------------------------------------------------------
%	EMPLOYMENT SECTION
%----------------------------------------------------------------------------------------
\section{Опыт работы}

\job
{Июнь 2021-}{Ноя 2023}
{\textbf{Место работы: }TOO "PAYDA ACCOUNTING DEVELOPMENT"}
{\textbf{Позиция}: Бэкенд разработчик}
\normalfont{\textbf{Обязанности}: Интеграция онлайн-кассы Ukassa.kz в Payda CRM для мэнэджемента чеков клиентов. Интеграция с порталом ЭСФ через протокол SOAP для упрощения генерации ЭСФ. Дальнейшая интеграция с порталом ЭСФ для массовой генерации АВР \\ \textbf{Технологии:} Django, DRF, FastAPI, SQLAlchemy}

\job
{Янв 2020-}{Окт 2020}
{\textbf{Место работы: }TOO "CodeBusters" }
{\textbf{Позиция}: Бэкенд стажировка. }
\normalfont{\textbf{Обязанности}: Участвовал в разработке платформы для школы-интерната IQanat \\ \textbf{Технологии}: Django, DRF }


%------------------------------------------------

%----------------------------------------------------------------------------------------
%	IT/COMPUTING SKILLS SECTION
%----------------------------------------------------------------------------------------

\section{Навыки}

\skillgroup{Web}
{
  Django, Django Rest Framework, Fastapi, SQLAlchemy, Postgresql
}

\skillgroup{Языки программирования}
{
  Python: asyncio, threading, GIL, generators\\
  Javascript: ES6
  Lua: для скриптинга и для конфигураций
}

\skillgroup{Базы данных, системы контроля версий и т.д.}
{
  Linux: пользователь, знание утилит как awk, grep, find, pipe\\
  Docker: знакомство с базовыми командами\\
  Git: знание на уровне решения merge conflict-ов\\
  Database: Postgresql
}

\skillgroup{Человеческие языки}
{
  Русский язык: разговорный\\
  Английский язык: на уровне чтения технической документации и мемов\\
  Казахский язык: базовый
}

%------------------------------------------------

\skillgroup{Среда разработки}
{
  SHELL: zsh\\
  IDE: Vim, Pycharm
}

\skillgroup{О себе}
{
  Играю шахматы, изучаю японский
}

\end{document}
